%%%%%%%%%%%%%%%%%%%%%%%%%%%%%%%%%%%%%%%%%
% After install and run FortiClient, choose REMOTE ACCESS > New VPN Connection > Choose SSL-VPN, and then fill in the information as follows.
 
% Connection Name: ADI
% Remote Gateway: secure.hksciencepark.org
 
% Click SAVE then Close to finish the VPN connection setup
% Please find the VPN account information as below:
 
% Username: lxe5wipauser14.vpn
% Password: PMdLucY3


% Please find the user account and credential below. Also, attached the onboarding guide and VPN -Forticlient install guide. 
% For VPN, please remind to activate within 24 hours and the login username and password is difference which will sent by separately email. 
% Linux login Username: lxe5wipauser14
% Login Password: Samuel1002DDL!
  
% Please save your data under the directory as follows  
% /scratch/lxe5wipa
% Please submit your job using this job queue. Kindly remind that you also need to input memory parameter for the job to run successfully 


% Best Regards,
% Thomas Lok

% As we advance our project, we have identified the need for Deepseek APIs to support the development of AI agents. We are requesting the purchase of an annual service for these APIs provided by Deepseek.Medium Length Professional CV
% LaTeX Template
% Version 2.0 (8/5/13)
%
% This template has been downloaded from:
% http://www.LaTeXTemplates.com
%
% Original author:
% Trey Hunner (http://www.treyhunner.com/)
%
% Important note:
% This template requires the resume.cls file to be in the same directory as the
% .tex file. The resume.cls file provides the resume style used for structuring the
% document.
%
%%%%%%%%%%%%%%%%%%%%%%%%%%%%%%%%%%%%%%%%%

%----------------------------------------------------------------------------------------
%	PACKAGES AND OTHER DOCUMENT CONFIGURATIONS
%----------------------------------------------------------------------------------------

\documentclass{resume} % Use the custom resume.cls style

\usepackage[left=0.75in,top=0.6in,right=0.75in,bottom=0.6in]{geometry} % Document margins

\name{Dongding Lin} % Your name
% \address{123 Broadway \\ City, State 12345} % Your address
% \address{123 Pleasant Lane \\ City, State 12345} % Your secondary addess (optional)
\address{Tel: (+86) 137 5006 5371 \\ Email: 22037064r@connect.polyu.hk} % Your phone number and email
%\address{Homepage: https://lindongding.com} % Your phone number and email

\begin{document}

%----------------------------------------------------------------------------------------
%	EDUCATION SECTION
%----------------------------------------------------------------------------------------

\begin{rSection}{Education}

{\bf The Hong Kong Polytechnic University} \hfill {\em Sep. 2022 - Present} \\ 
PhD candidate in Department of Computing \hfill {\em \bf{}} \\
{\bf Sun Yat-sen University} \hfill {\em Sep. 2017 - Jul. 2020} \\ 
M.Eng. in Computer Technology \hfill {\em \bf{GPA: 3.9/4.0}} \\
{\bf Sun Yat-sen University} \hfill {\em Sep. 2013 - Jul. 2017} \\ 
B.Eng. in Software Engineering \hfill {\em \bf{GPA: 3.8/4.0  Rank:37/433}} \\

\end{rSection}

%----------------------------------------------------------------------------------------
%	WORK EXPERIENCE SECTION
%----------------------------------------------------------------------------------------
\begin{rSection}{PUBLICATIONS}
{\bf{Dongding Lin}}, Jian Wang, Chak Tou Leong, Wenjie Li. SCREEN: A Benchmark for Situated Conversational Recommendation. (ACM MM 2024). \\
Jian Wang, {\bf{Dongding Lin}}, Wenjie Li. Target-constrained Bidirectional Planning for Generation of Target-oriented Proactive Dialogue. (TOIS 2024). \\
Jian Wang, Chak Tou Leong, Jiashuo Wang, {\bf{Dongding Lin}}, Wenjie Li, Xiao-Yong Wei. Target-constrained Bidirectional Planning for Generation of Target-oriented Proactive Dialogue. (TOIS 2024). \\
Jian Wang, Yi Cheng, {\bf{Dongding Lin}}, Chak Tou Leong, Wenjie Li. Target-oriented proactive dialogue systems with personalization: Problem formulation and dataset curation. (TOIS 2024). \\
{\bf{Dongding Lin*}}, Jian Wang*, Wenjie Li. COLA: Improving Conversational Recommender Systems by Collaborative Augmentation. (*: Equal Contribution) (AAAI 2023). \\
Jian Wang*, {\bf{Dongding Lin*}}, Wenjie Li. Dialogue Planning via Brownian Bridge Stochastic Process for Goal-directed Proactive Dialogue. (*: Equal Contribution) (ACL 2023 Findings). \\
Jian Wang, {\bf{Dongding Lin}}, Wenjie Li. A target-driven planning approach for goal-directed dialog systems. (TNNLS 2023).
{\bf{Dongding Lin}}, Jian Wang, Wenjie Li. Target-guided Knowledge-aware Recommendation Dialogue System: An Empirical Investigation. (KaRS@RecSys 2021).\\
Fenfang Xie, {\bf{Dongding Lin}}, et.al. Personalized Service Recommendation With Mashup Group Preference in Heterogeneous Information Network. (IEEE Access 2019).\\
Fenfang Xie, {\bf{Dongding Lin}}, et.al. Poster: Group Preference based API recommendation via heterogeneous information network. (ICSE 2018).\\

\end{rSection}


\begin{rSection}{Research Experiences}

\begin{rSubsection}{NLP Group, PolyU}{\em{Dec. 2022 - Present}}{Situated Conversational Recommendation}{Supervisor: Prof. Wenjie, Li}
\item Investigating techniques related to large-scale pre-trained language models and multi-modal dialogue system with the goal of increasing user engagement in our dialogue system. By leveraging multiple types of information, we aim to create a more immersive experience for the user.
\end{rSubsection}

\begin{rSubsection}{NLP Group, PolyU}{\em{July. 2021 - Dec. 2022}}{Conversational Recommender System}{Supervisor: Prof. Wenjie, Li}
\item Investigated the problem of target-guided knowledge-aware recommendation dialogue and design a dialogue generation system to make high-quality recommendations through interactive conversations proactively and naturally. This work has been accepted by the 3rd Workshop of Knowledge-aware and Conversational Recommender Systems {\bf{(KaRS 2021)}} and the 37th AAAI conference on Artificial Intelligence {\bf{(AAAI 2023)}}.
\end{rSubsection}

\begin{rSubsection}{Collective Intelligence Systems Lab, SYSU}{\em{Feb. 2018 - Jul. 2020}}{Machine Reading Comprehension}{Supervisor: Prof. Rong Pan}

\item Proposed a Hierarchical global-aware Information
Transmission mechanism, which makes use of the interaction between local-aware information and global-aware information to improve the effectiveness of the model.
\item Employed a Memory Flow mechanism. The memory information is updated by reading and writing operations on the memory module so that the model can not only store the information of historical questions and answers but also modify the stored information according to the current question and answer.
%\item  Combing the Delta information flow module to further introduce time series information.
% \item Exploring separating the passage into different graph nodes and using the Graph Neural Network to establish node contact.
\item The experimental results on CoQA datasets showed that our model obtained competitive results compared to the existing methods. Our ablation analysis results demonstrated the effectiveness of each step. This work is also the content of my postgraduate thesis.

\end{rSubsection}

%------------------------------------------------

\begin{rSubsection}{NLP Group at SIAT, Chinese Academy of Sciences}{{\em{Apr. 2019 - Oct. 2019}}}{Conversational Machine Reading Comprehension}{Supervisor: Prof. Min Yang}
\item Proposed a Hierarchical Conversation Flow Transition and Reasoning (HCFTR) model for conversational machine reading comprehension. One multi-flow transition mechanism is designed to integrate the global-aware information flow transition and make dynamic reasoning. One multi-level flow-context attention mechanism is developed to fuse multiple levels of hierarchical fine-grained representations and perform advanced reasoning. 
\item Experimental results on two benchmark datasets show that our model received competitive performance.
\end{rSubsection}

%--------------------------------------------------
\begin{rSubsection}{Mobile Internet and Financial Big Data Lab, SYSU}{{\em{Jul. 2016 - Jul. 2017}}}{Service-Oriented Computing}{Supervisor: Prof. ZiBin Zheng}
\item Designed a Group Preference based API Recommendation System. The Ranking Recommendation System used the Bayesian Personalized Ranking Model, recommended appropriate APIs for Mashup (a new type of service in internet by integrating multiple sources or functions on the network). A series of experiments conducted on a real-world dataset demonstrate our proposed approach outperforms other baseline approaches.
\item This work was also my undergraduate thesis’s topic, given the award of the SYSU {\bf{Excellent Graduation Thesis}} and accepted by International Conference on Software Engineering {\bf{(ICSE)-2018}} and {\bf{IEEE Access-2019}}.
\end{rSubsection}

\end{rSection}




%
\begin{rSection}{Honors/Awards}
Ranked {\em{4/750}} in 2021 Language and Intelligence Challenge (Baidu). \hfill{{\em{June 2021}}}\\
The Third Prize Postgraduate Scholarship of Sun Yat-sen University. \hfill{{\em{2018-2019}}}\\
Ranked {\em{4/119}} in Materials Quality Prediction Kaggle Competition \hfill{{\em{April-July 2018}}}\\
The Second Prize Postgraduate Scholarship of Sun Yat-sen University. \hfill{{\em{2017-2018}}}\\
{\bf{Outstanding Graduates Awards}} of Sun Yat-sen University ({\bf{Top 3\%}} awarded) \hfill{{\em{June 2017}}}\\
{\bf{Excellent Graduation Thesis}} of Sun Yat-sen University ({\bf{Top 3\%}} awarded) \hfill{{\em{June 2017}}}\\
The {\bf{First}} Prize Scholarship of Sun Yat-sen University ({\bf{Top 5\%}} awarded) \hfill{{\em{2016-2017}}}\\
The {\bf{First}} Prize Scholarship of Sun Yat-sen University ({\bf{Top 5\%}} awarded) \hfill{{\em{2015-2016}}}\\
The Third Prize Scholarship of Sun Yat-sen University  \hfill{{\em{2014-2015}}}\\
\end{rSection}

%

%
% \begin{rSection}{Teaching Experience}

% Teaching assistant for {\em{Mobile Web Application Design}}, SYSU \hfill{{\em{Fall 2016}}}\\
% Teaching assistant for {\em{Artificial Intelligence}}, SYSU \hfill{{\em{Fall 2016}}}\\
% Teaching assistant for {\em{Operation System}}, SYSU \hfill{{\em{Spring 2016}}}\\
% Teaching assistant for {\em{Digital Signal Processing and Control}}, SYSU \hfill{{\em{Spring 2016}}}\\
% Teaching assistant for {\em{Computer Organization and Design}}, SYSU \hfill{{\em{Spring 2016}}}\\

% \end{rSection}

%

%----------------------------------------------------------------------------------------
%	TECHNICAL STRENGTHS SECTION
%----------------------------------------------------------------------------------------

\begin{rSection}{Technical Strengths}

\begin{tabular}{ @{} >{\bfseries}l @{\hspace{6ex}} l }
Interest & NLP, Dialogue System, CRS, Pretraining, Prompting, etc\\
Programming Languages & Python, C/C++, Java, Latex, JavaScript, MySQL,  MATLAB\\
DL Framework & Pytorch, Tensorflow \\
English & CET4, CET6, IELTS(6.5)
\end{tabular}

\end{rSection}
%----------------------------------------------------------------------------------------
%	EXAMPLE SECTION
%----------------------------------------------------------------------------------------

%\begin{rSection}{Section Name}

%Section content\ldots

%\end{rSection}

%----------------------------------------------------------------------------------------

\end{document}
