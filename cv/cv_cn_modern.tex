\documentclass[11pt,a4paper,fontset=fandol]{ctexart}

\usepackage[left=1.35cm,right=1.35cm,top=1.1cm,bottom=1.1cm]{geometry}
\usepackage{tabularx}
\usepackage{array}
\usepackage{enumitem}
\usepackage[dvipsnames]{xcolor}
\usepackage[hidelinks]{hyperref}
\usepackage{fontawesome5}

\pagestyle{empty}
\setlength{\parindent}{0pt}
\setlength{\parskip}{2.1pt}
\setlength{\emergencystretch}{2em}
\renewcommand{\arraystretch}{1.08}
\setlist[itemize]{leftmargin=1.35em,itemsep=1.6pt,topsep=1.4pt,parsep=0pt,partopsep=0pt}

\definecolor{Accent}{RGB}{38,115,220}
\definecolor{RuleGray}{RGB}{140,145,155}
\definecolor{BodyText}{RGB}{30,36,48}

\newcommand{\Section}[1]{%
    \vspace{0.50em}
    {\color{Accent}\bfseries\large #1}\par
    \vspace{0.26em}
    {\color{RuleGray}\hrule height 0.8pt}
    \vspace{0.42em}
}

\newcommand{\Entry}[4]{%
\begin{tabularx}{\linewidth}{@{}Xr@{}}
\textbf{#1} & \textit{#2}\\
{\small #3} & {\small #4}
\end{tabularx}\vspace{0.16em}
}

\begin{document}
\color{BodyText}

\begin{center}
{\fontsize{23}{23}\selectfont\bfseries\color{Accent} 林东定 \quad DONGDING LIN}\\[3pt]
{\small
\faMapMarker*\ 香港 \ \textbullet\ 
\faPhone\ (+86)~137~5006~5371 \ \textbullet\ 
\faEnvelope\ \href{mailto:22037064r@connect.polyu.hk}{22037064r@connect.polyu.hk}\\[2pt]
\faGithub\ \href{https://github.com/DongdingLin}{github.com/DongdingLin}
\ \textbullet\ 
\faGraduationCap\ \href{https://scholar.google.com/citations?user=JM4i0R8AAAAJ}{Google Scholar}
\ \textbullet\ 
\textbf{求职方向:}大模型算法工程师 / LLM 研究工程师
}
\end{center}

\Section{个人概述}
香港理工大学计算学系博士研究生,研究方向聚焦 \textbf{LLM 推理、对话系统、多模态 LLM 与会话推荐}。具备从任务定义、数据构建、模型设计到实验评测的完整研发能力,持续在 ACM MM、ACL、AAAI、TOIS、TNNLS 等产出成果。

\Section{教育背景}
\Entry{香港理工大学}{2022.09 -- 至今}{计算学系博士研究生(NLP Group)}{香港}
\Entry{中山大学}{2017.09 -- 2020.07}{计算机技术硕士,GPA 3.9/4.0}{广州}
\Entry{中山大学}{2013.09 -- 2017.07}{软件工程学士,GPA 3.8/4.0(排名 37/433)}{广州}

\Section{实习与研究经历}
\Entry{华为香港研究中心(HKRC)Fermat Lab}{2025.08 -- 至今}{Research Intern(大模型相关)}{香港}
\begin{itemize}
    \item \textbf{无线网络边缘侧在线学习与数据剪枝:}面向端侧算力受限场景,构建端云协同轻量训练方案;结合混合 OOD 过滤、蒸馏难例挖掘与 GPU 加速 K-Means 去冗余,在仿真中实现 \textbf{4x} 算力下降(75\%)且精度保持率 \textbf{$>$97\%}。
    \item \textbf{神经符号数学推理数据工厂:}搭建“LLM 语义解析 + Mathematica 符号验证”闭环,保障合成数据逻辑一致性与标签准确性;AIME/HMMT 风格题泛化成功率由 \textbf{7.8\%} 提升至 \textbf{63.95\%},Math500 新题正确率提升至 \textbf{99\%}。
    \item \textbf{真实对话数据清洗与语义对齐:}基于句向量与聚类替代规则清洗,去除 ASR 噪声、无效拒答与长尾离群样本,筛选高价值 I.I.D. 黄金数据,提升对话模型在真实场景的鲁棒性与响应质量。
\end{itemize}

\Entry{香港理工大学 NLP Group}{2022.12 -- 至今}{Research Assistant / PhD Research}{香港}
\begin{itemize}
    \item 研究方向:情境会话推荐(SCR)、LLM 推理、多模态理解。
    \item 负责模型方案设计、实验评测与论文产出,形成多篇顶会/期刊成果。
    \item 主导 benchmark 构建与基线系统实现,推动可复用研究流程沉淀。
\end{itemize}

\Entry{香港理工大学 NLP Group}{2021.07 -- 2022.12}{Research Assistant}{香港}
\begin{itemize}
    \item 研究目标导向会话系统与会话推荐,搭建可复用的对话规划生成框架。
\end{itemize}

\Entry{中山大学集体智能系统实验室}{2018.02 -- 2020.07}{Research Assistant}{广州}
\begin{itemize}
    \item 研究机器阅读理解,提出分层信息传递与记忆流机制并验证其有效性。
\end{itemize}

\Section{核心项目(LLM)}
\textbf{SCREEN:情境会话推荐基准} \hfill \textit{ACM MM 2024}
\begin{itemize}
    \item 提出情境会话推荐任务设定,构建 \textbf{20k+ 对话、1.5k 场景} benchmark。
    \item 设计子任务与基线实验,为真实交互场景下的大模型能力评估提供标准化支撑。
\end{itemize}

\textbf{MIDI-Tuning:多轮对话一致性高效调优} \hfill \textit{ACL 2024}
\begin{itemize}
    \item 提出角色建模驱动的多轮调优方法,提升对话一致性与可控性。
\end{itemize}

\textbf{TRIP:目标约束双向规划} \hfill \textit{TOIS 2024}
\begin{itemize}
    \item 通过 look-ahead / look-back 双向规划策略,提升目标导向对话生成质量。
\end{itemize}

\Section{代表论文(Selected)}
\begin{itemize}
    \item \textbf{Dongding Lin}, Jian Wang, Chak Tou Leong, Wenjie Li. SCREEN: A Benchmark for Situated Conversational Recommendation. \textit{ACM MM 2024}.
    \item Jian Wang, \textbf{Dongding Lin}, Wenjie Li. Target-constrained Bidirectional Planning for Generation of Target-oriented Proactive Dialogue. \textit{TOIS 2024}.
    \item Jian Wang, Chak Tou Leong, Jiashuo Wang, \textbf{Dongding Lin}, Wenjie Li, Xiao-Yong Wei. Instruct once, chat consistently in multiple rounds. \textit{ACL 2024}.
    \item Jian Wang, Yi Cheng, \textbf{Dongding Lin}, et al. Target-oriented proactive dialogue systems with personalization. \textit{EMNLP 2023}.
    \item \textbf{Dongding Lin*}, Jian Wang*, Wenjie Li. COLA: Improving Conversational Recommender Systems by Collaborative Augmentation. \textit{AAAI 2023}.
\end{itemize}

\Section{学术服务与教学}
\begin{itemize}
    \item 审稿服务:ACL Rolling Review(ARR)、ACL、EMNLP、ACM MM。
    \item 助教课程:\textit{自然语言处理}(2024/25 S2,2023/24 S2)、\textit{数据结构与数据库系统}(2024/25 S1)、\textit{移动计算}(2023/24 S1)。
\end{itemize}

\Section{技术栈}
\begin{tabularx}{\linewidth}{@{}>{\bfseries}p{2.9cm}X@{}}
编程语言 & Python, C/C++, Java, JavaScript, SQL, MATLAB \\
框架工具 & PyTorch, TensorFlow, Hugging Face, Scikit-learn, Linux, Git, LaTeX \\
能力关键词 & LLM 训练与推理、Prompt/Agent、对话系统、会话推荐、多模态理解
\end{tabularx}

\Section{荣誉奖项}
\begin{itemize}
    \item 百度 2021 语言与智能技术竞赛:\textbf{4/750}
    \item Kaggle 材料质量预测竞赛:\textbf{4/119}
    \item 中山大学优秀毕业生(前 3\%)、优秀毕业论文(前 3\%)
    \item 中山大学本科与研究生阶段多次一/二/三等奖学金
\end{itemize}

\end{document}
