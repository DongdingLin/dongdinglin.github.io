\documentclass[11pt,a4paper,fontset=fandol]{ctexart}

\usepackage[left=1.35cm,right=1.35cm,top=1.1cm,bottom=1.1cm]{geometry}
\usepackage{tabularx}
\usepackage{array}
\usepackage{enumitem}
\usepackage[dvipsnames]{xcolor}
\usepackage[hidelinks]{hyperref}
\usepackage{fontawesome5}

\pagestyle{empty}
\setlength{\parindent}{0pt}
\setlength{\parskip}{2.1pt}
\setlength{\emergencystretch}{2em}
\renewcommand{\arraystretch}{1.08}
\setlist[itemize]{leftmargin=1.35em,itemsep=1.6pt,topsep=1.4pt,parsep=0pt,partopsep=0pt}

\definecolor{Accent}{RGB}{38,115,220}
\definecolor{RuleGray}{RGB}{140,145,155}
\definecolor{BodyText}{RGB}{30,36,48}

\newcommand{\Section}[1]{%
    \vspace{0.50em}
    {\color{Accent}\bfseries\large #1}\par
    \vspace{0.26em}
    {\color{RuleGray}\hrule height 0.8pt}
    \vspace{0.42em}
}

\newcommand{\Entry}[4]{%
\begin{tabularx}{\linewidth}{@{}Xr@{}}
\textbf{#1} & \textit{#2}\\
{\small #3} & {\small #4}
\end{tabularx}\vspace{0.16em}
}

\newcommand{\EduOneLine}[5][0.18em]{%
\begin{tabularx}{\linewidth}{@{}>{\bfseries}p{3.2cm} >{\raggedright\arraybackslash}X r@{}}
#2 & {\small #4} & {\small \textit{#3} \ \textbullet\ #5}
\end{tabularx}\vspace{#1}
}

\begin{document}
\color{BodyText}

\begin{center}
{\fontsize{23}{23}\selectfont\bfseries\color{Accent} 林东定 \quad LIN DONGDING}\\[3pt]
{\small
\faMapMarker*\ 香港 \ \textbullet\ 
\faPhone\ (+86)~137~5006~5371 \ \textbullet\ 
\faEnvelope\ \href{mailto:22037064r@connect.polyu.hk}{22037064r@connect.polyu.hk}\\[2pt]
\faGithub\ \href{https://github.com/DongdingLin}{github.com/DongdingLin}
\ \textbullet\ 
\faGraduationCap\ \href{https://scholar.google.com/citations?user=JM4i0R8AAAAJ}{Google Scholar}
\ \textbullet\ 
\textbf{求职方向:}大模型算法工程师
}
\end{center}

\Section{个人概述}
香港理工大学电子计算学系博士研究生,研究聚焦 \textbf{LLM 推理、对话系统、多模态大模型与情境会话推荐}。
熟悉从任务定义、数据构建、模型设计到实验评测的完整研发流程,并具备工程落地与迭代优化能力,相关工作发表于 ACM MM、ACL、AAAI、TOIS、TNNLS 等。

\Section{教育背景}
\EduOneLine{香港理工大学}{2022.09 -- 至今}{电子计算学系博士研究生(NLP Group,GPA 4.18/4.3)}{香港,中国}
\EduOneLine{中山大学}{2017.09 -- 2020.07}{计算机技术硕士(推免/保研),GPA 3.9/4.0}{广州,中国}
\EduOneLine[-1.4em]{中山大学}{2013.09 -- 2017.07}{软件工程学士,GPA 3.8/4.0(排名 37/433)}{广州,中国}

\Section{实习与研究经历}
\Entry{华为香港研究中心(HKRC)Fermat Lab}{2025.08 -- 至今}{Research Intern,优秀实习生}{香港}
\begin{itemize}
    \item \textbf{无线在线训练数据筛减(基站资源优化):}
    面向在线训练场景,设计本地数据筛选与分布外(OOD)过滤策略;结合蒸馏式判别与 GPU-KMeans 聚类,在基本不损性能的前提下将训练算力 \textbf{降低 4x},性能衰减控制在 \textbf{≤3\%},并提交技术报告和相应的代码实现,形成可复用配置与日志闭环,支持快速回滚与复现实验。
    \item \textbf{数学题数据自动泛化 + 形式化校验(Math500/竞赛/K12):} 搭建自动流水线:变量抽取 $\rightarrow$ 参数泛化 $\rightarrow$ Mathematica 求解与一致性校验,并输出结构化标注以支持多题型扩展,支持批量生成;将 Math500 泛化率从 \textbf{42.74\% 提升至 58.6\%},新题准确率从 \textbf{87.4\% 提升至 99\%};竞赛题泛化率从 \textbf{7.8\% 提升至 63.95\%};在 K12 多题型场景引入多路泛化策略,将整体泛化率提升至 \textbf{42\%};核心能力封装为 \textbf{SDK} 交付落地。
    \item \textbf{同分布(IID)对话数据筛选:}为模拟真实用户提问并用于新模型效果评测,构建同分布且高质量的对话数据;以句向量检索与聚类结合规则清洗,剔除ASR噪声、拒答与离群样本,沉淀高价值 IID 数据。

\end{itemize}

\Entry{香港理工大学 NLP Group}{2022.12 -- 至今}{PhD Research}{香港}
\begin{itemize}
    \item \textbf{情境会话推荐(SCR)建模与评测:}围绕场景、上下文与用户偏好,梳理交互链路并定义任务设定与评测指标;实现数据处理、训练与评测的端到端流水线,支撑系列论文与可复现实验。
    \item \textbf{SiPeR:场景迁移 + 贝叶斯偏好推断:}面向 SCR 的“去哪/选什么(Where/What)”决策,同时建模场景迁移与目标场景预测,估计当前场景是否满足需求并在必要时引导迁移到更合适场景;利用多模态大模型 likelihood 做贝叶斯逆推断以刻画隐式偏好,在 SIMMC 2.1 与 SCREEN 上平均提升 \textbf{10.9\%/10.6\%}。

    \item \textbf{Re2A:Rubric 驱动的偏好推理与双对齐生成:}将 SCR 设计为 reason-then-align 流程:用自动 Rubrics 提供多维奖励信号,实现\textbf{无需人工标注}的显式偏好状态推理;
    再用 Preference-conditioned DPO 同时对齐“偏好满足 + 场景一致”,在 SIMMC 2.1 上将幻觉率从 \textbf{25.4\% 降至 5.2\%}、弱匹配率从 \textbf{22.6\% 降至 7.4\%}。
\end{itemize}

\Entry{香港理工大学 NLP Group}{2021.07 -- 2022.12}{Research Assistant}{香港}
\begin{itemize}
    \item \textbf{目标导向对话规划与生成(框架工程化):}
    围绕“目标约束、对话状态与用户偏好”设计对话规划与生成方案,引入 look-ahead/反馈修正等策略提升多轮可控性与目标达成;并将数据预处理、状态/意图建模、规划器与生成器模块化封装,统一训练/推理接口与评测脚本,支持多策略快速对比与复现实验。
\end{itemize}

\Entry{中山大学CIS实验室}{2018.02 -- 2020.07}{Research Assistant}{广州}
\begin{itemize}
    \item \textbf{机器阅读理解与多跳推理(层级传播 + 记忆流):}
    面向长文档证据聚合与跨句关系建模,提出分层信息传递与记忆更新机制增强证据追踪与信息保真,搭建训练/评测流水线并开展系统性消融与误差分析,沉淀可复现实验代码与配置。
\end{itemize}

\Section{代表论文(Selected)}
\begin{itemize}
    \item \textbf{Dongding Lin}, Jian Wang, Chak Tou Leong, Wenjie Li. SCREEN: A Benchmark for Situated Conversational Recommendation. \textit{ACM MM 2024}.
    \item Jian Wang, \textbf{Dongding Lin}, Wenjie Li. Target-constrained Bidirectional Planning for Generation of Target-oriented Proactive Dialogue. \textit{TOIS 2024}.
    \item Jian Wang, Chak Tou Leong, Jiashuo Wang, \textbf{Dongding Lin}, Wenjie Li, Xiao-Yong Wei. Instruct once, chat consistently in multiple rounds. \textit{ACL 2024}.
    \item Jian Wang, Yi Cheng, \textbf{Dongding Lin}, et al. Target-oriented proactive dialogue systems with personalization. \textit{EMNLP 2023}.
    \item \textbf{Dongding Lin*}, Jian Wang*, Wenjie Li. COLA: Improving Conversational Recommender Systems by Collaborative Augmentation. \textit{AAAI 2023}.
\end{itemize}

\Section{学术服务与教学}
\begin{itemize}
    \item 审稿服务:ACL Rolling Review(ARR)、ACL、EMNLP、ACM MM。
    \item 助教课程:\textit{自然语言处理}(2024/25 S2,2023/24 S2)、\textit{数据结构与数据库系统}(2024/25 S1)、\textit{移动计算}(2023/24 S1)。
\end{itemize}

\Section{技术栈}
\begin{tabularx}{\linewidth}{@{}>{\bfseries}p{2.9cm}X@{}}
编程语言 & Python, C/C++, Java, JavaScript, SQL, MATLAB \\
深度学习工具 & PyTorch, TensorFlow, Hugging Face, Scikit-learn \\
兴趣研究方向 & LLM 训练与推理、对话系统、会话推荐、多模态理解
\end{tabularx}

\Section{荣誉奖项}
\begin{itemize}
    \item 百度 2021 语言与智能技术竞赛:\textbf{4/750}
    \item Kaggle 材料质量预测竞赛:\textbf{4/119}
    \item 中山大学优秀毕业生(前 3\%)、优秀毕业论文(前 3\%)
    \item 中山大学本科与研究生阶段多次一/二/三等奖学金
\end{itemize}

\end{document}
