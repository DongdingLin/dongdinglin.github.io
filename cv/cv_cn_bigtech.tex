\documentclass[11pt,a4paper,fontset=fandol]{ctexart}

\usepackage[left=1.6cm,right=1.6cm,top=1.2cm,bottom=1.2cm]{geometry}
\usepackage{array}
\usepackage{enumitem}
\usepackage{hyperref}
\usepackage{xcolor}
\usepackage{titlesec}

\hypersetup{
    colorlinks=true,
    urlcolor=[RGB]{26,76,149},
    linkcolor=black
}

\pagestyle{empty}
\setlength{\parindent}{0pt}
\setlist[itemize]{leftmargin=1.4em, itemsep=2pt, topsep=2pt}

\titleformat{\section}{\large\bfseries}{}{0em}{}[\vspace{-0.3em}\titlerule]
\titlespacing*{\section}{0pt}{0.8em}{0.5em}

\newcommand{\ResumeEntry}[4]{
    \textbf{#1} \hfill {\small #2} \\
    {\small #3} \hfill {\small #4} \\
}

\begin{document}

{\LARGE\textbf{林东定 \quad Dongding Lin}} \hfill {\small 意向岗位:大模型算法工程师 / LLM 研究工程师} \\
{\small 手机:(+86) 137 5006 5371 \quad 邮箱:22037064r@connect.polyu.hk \quad 地点:香港} \\
{\small GitHub:\url{https://github.com/DongdingLin} \quad Google Scholar:\url{https://scholar.google.com/citations?user=JM4i0R8AAAAJ}}

\section*{个人优势}
\begin{itemize}
    \item 聚焦 \textbf{LLM 对话系统、会话推荐、多模态 LLM、推理增强},研究与工程结合紧密。
    \item 主导构建 SCREEN 基准:覆盖 \textbf{1.5k 场景、20k+ 多轮对话},支撑情境会话推荐任务评测。
    \item 在 ACM MM、ACL、AAAI、TOIS、TNNLS 等发表多篇论文,具备从问题定义到实验落地的完整闭环能力。
\end{itemize}

\section*{教育背景}
\ResumeEntry{香港理工大学(The Hong Kong Polytechnic University)}{2022.09 -- 至今}{计算学系 \textbf{博士研究生}(NLP Group)}{导师:Wenjie Li}
\ResumeEntry{中山大学}{2017.09 -- 2020.07}{计算机技术 \textbf{硕士}}{GPA:3.9/4.0}
\ResumeEntry{中山大学}{2013.09 -- 2017.07}{软件工程 \textbf{学士}}{GPA:3.8/4.0,排名 37/433}

\section*{实习与研究经历}
\ResumeEntry{华为香港研究中心(HKRC), Fermat Lab}{2025.08 -- 至今}{Research Intern(大模型相关)}{}
\begin{itemize}
    \item 参与无线优化与数学推理方向的数据构建,支持大模型训练/评测场景的数据生产。
    \item 面向工业问题设计任务数据与评估方案,提升数据可用性与迭代效率。
\end{itemize}

\ResumeEntry{香港理工大学 NLP Group}{2022.12 -- 至今}{Research Assistant / PhD Research}{}
\begin{itemize}
    \item 研究方向:情境会话推荐(SCR)、LLM 推理、多模态 LLM。
    \item 负责模型方案设计、数据构建、实验评测与论文写作,持续产出顶会/期刊成果。
\end{itemize}

\ResumeEntry{香港理工大学 NLP Group}{2021.07 -- 2022.12}{Research Assistant}{}
\begin{itemize}
    \item 研究目标导向会话系统与会话推荐,围绕对话规划与生成建立可复用技术框架。
\end{itemize}

\section*{核心项目(大模型方向)}
\textbf{SCREEN: Situated Conversational Recommendation Benchmark(ACM MM 2024)} \\
\begin{itemize}
    \item 提出情境会话推荐任务设定,使用多模态大模型角色扮演生成高质量训练/评测数据。
    \item 构建 20k+ 对话、1.5k 场景基准,定义子任务并完成多模型系统评测。
\end{itemize}

\textbf{MIDI-Tuning: Multi-round Interactive Dialogue Tuning(ACL 2024)} \\
\begin{itemize}
    \item 设计多轮对话高效调优框架,分角色建模 agent/user,提升多轮一致性与稳定性。
\end{itemize}

\textbf{TRIP: Target-constrained Bidirectional Planning(TOIS 2024)} \\
\begin{itemize}
    \item 提出双向规划机制(look-ahead / look-back),增强目标导向对话的规划质量与可控性。
\end{itemize}

\section*{论文与成果(Selected)}
\begin{itemize}
    \item \textbf{Dongding Lin}, Jian Wang, Chak Tou Leong, Wenjie Li. SCREEN: A Benchmark for Situated Conversational Recommendation. \textit{ACM MM 2024}.
    \item Jian Wang, \textbf{Dongding Lin}, Wenjie Li. Target-constrained Bidirectional Planning for Generation of Target-oriented Proactive Dialogue. \textit{TOIS 2024}.
    \item Jian Wang, Chak Tou Leong, Jiashuo Wang, \textbf{Dongding Lin}, Wenjie Li, Xiao-Yong Wei. Instruct once, chat consistently in multiple rounds. \textit{ACL 2024}.
    \item \textbf{Dongding Lin*}, Jian Wang*, Wenjie Li. COLA: Improving Conversational Recommender Systems by Collaborative Augmentation. \textit{AAAI 2023}.
    \item Jian Wang*, \textbf{Dongding Lin*}, Wenjie Li. Dialogue Planning via Brownian Bridge Stochastic Process for Goal-directed Proactive Dialogue. \textit{ACL Findings 2023}.
\end{itemize}

\section*{技术栈}
\begin{tabular}{>{\bfseries}p{2.8cm}p{12.7cm}}
编程语言 & Python, C/C++, Java, JavaScript, SQL, MATLAB \\
框架与工具 & PyTorch, TensorFlow, Hugging Face, Scikit-learn, Linux, Git, LaTeX \\
方向关键词 & LLM 训练与推理、Prompt/Agent、对话系统、会话推荐、多模态理解 \\
语言能力 & 中文(母语)、英文(CET-4/6,IELTS 6.5) \\
\end{tabular}

\section*{荣誉奖项}
\begin{itemize}
    \item 百度 2021 语言与智能技术竞赛:\textbf{4/750}
    \item Kaggle 材料质量预测竞赛:\textbf{4/119}
    \item 中山大学优秀毕业生(前 3\%)、优秀毕业论文(前 3\%)
    \item 中山大学研究生奖学金(二等奖、三等奖)
\end{itemize}

\end{document}
